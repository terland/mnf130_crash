\documentclass{beamer}
\usepackage{hyperref}
\usepackage{amsfonts}
\usepackage[T1]{fontenc}
\usepackage{lmodern}
\usepackage[utf8]{inputenc}
\usepackage[T1]{fontenc}
\usepackage{lmodern} % load a font with all the characters
\usepackage{amssymb}
\usepackage{graphicx}
\usepackage{appendix}
\usepackage{tikz}
\usepackage{algorithm}
\usepackage{algpseudocode}
\usepackage{pifont}
\usepackage{amsthm}
\usepackage{hyperref}
\usepackage{algorithm}
\usepackage{algpseudocode}
\usepackage[autostyle]{csquotes}  
\graphicspath{ {img/} }
\usepackage{default}

\usetheme{Montpellier}


\title[MNF130 Oppsummering]{MNF130 Oppsummeringsforelesning}

\author{Håvard Utne Terland}
\institute[UiB]
{
	Universitetet i Bergen \\
	\medskip
\textit{Havard.Terland@student.uib.no}

}
\date{1. Juni, 2018}

\begin{document}
\begin{frame}
	\titlepage
\end{frame}


\begin{frame}
	\frametitle{Plan}
	Se på oppgaver fra de to forrige eksamenene.
	\begin{itemize}
		\item Formell logikk
		\item Mengder og funksjoner
		\item Primtallsfinning, gcd, lcm
		\item Induksjonsoppgaver fra eksamener
		\item Litt telling (kombinatorikk)
		\item Relasjoner
	\end{itemize}
\end{frame}

%\begin{frame}
%	\frametitle{Oversikt}
%	\tableofcontents
%\end{frame}

\section{Logikk}

\subsection{Utsagnslogikk (propositional logic)}

\begin{frame}
	\frametitle{Symbolsk logikk - 1}
	\begin{itemize}
		\item Proposisjoner: $P,Q,R,S,T,\dots$ representerer setninger som nødvendigvis er sann eller ikke sann (usann).
		\item Vi har symboler $\vee,\wedge,\neg \text{ og } \rightarrow$ som vi kan bruke.
		\item Sannhetstabeller: Vi kan manuelt sjekke om en setning er sann, gitt at vi vet om atomene er sann eller ikke.
		 
	\end{itemize}


\end{frame}

\begin{frame}
\frametitle{Symbolsk logikk - Noen viktige begrep}
\begin{itemize}
	\item $P \equiv Q$ - P \textbf{ekvivalent} med Q - betyr at P er sann nøyaktig når Q er sann og usann nøyaktig når Q er usann. Kan sjekkes i sannhetstabeller, eller med andre regler.
	\item \textit{Ekstremt} viktig: $P \rightarrow Q \equiv \neg P \vee Q$
	\item $\wedge,\vee$ er \textbf{kommutativ} og \textbf{assosiativ}. For eks får vi \[P \vee Q \equiv Q \vee P\] \[(P \wedge Q) \wedge R \equiv P \wedge (Q \wedge R)\].
	\item Noe er en \textbf{tautologi} hvis det i alle rader i sannhetstabellen er T. \textbf{Kontradiksjon} hvis alle rader er F.
\end{itemize}
\end{frame}

\begin{frame}
\begin{table}[]
	\centering
	\caption{$\rightarrow$ er IKKE assosiativ!}
	\label{my-label}
	\begin{tabular}{|l|l|l|l|}
		\hline
		$P$ & $P \rightarrow P$ & $(P \rightarrow P) \rightarrow P$ & $P \rightarrow (P \rightarrow P)$ \\ \hline
		T& T&  T&T  \\ \hline
		F& T&  F&T \\ \hline
	\end{tabular}
\end{table}

\begin{table}[]
	\centering
	\caption{$\rightarrow$ er IKKE kommutativ!}
	\label{my-label}
	\begin{tabular}{|l|l|l|l|}
		\hline
		$P$ & $Q$ & $P \rightarrow Q$ & $Q \rightarrow P$ \\ \hline
		T& T&  T&T  \\ \hline
		T& F&  F&T \\ \hline
		F& T&  T&F \\ \hline
		F& F&  T&T \\ \hline
	\end{tabular}
\end{table}


\end{frame}

\begin{frame}
Se på V16, 3b): Vis at $[(p \rightarrow q) \wedge p] \rightarrow q$ er en tautologi (benytt ekvivalente utrykk).
\end{frame}


\subsection{Kvantorer}

\begin{frame}
	\frametitle{Predikater}
	En proposisjon er \textit{konstant}. Ett predikat er en variabel setning; sannhetsverdi avhenger av input $P$ er predikatet, $P(x)$ er enten sann eller usann, avhengig av $x$. \newline
\end{frame}

\begin{frame}
\frametitle{Kvantorer: \ensuremath{\forall,\exists} osv.}
\begin{itemize}
	\item Vi jobber alltid i ett univers. For eksempel $\mathbb{N}$, eller kanskje noe så enkelt som $\{1,2,3\}$. 
	\item $\forall x(2x \geq x)$ er sann i universene $\mathbb{N}$ og $\{2,4,6,8,\dots\}$, men ikke om universet er $\mathbb{Z}$
	\item Rekkefølge på kvantorene er \textit{helt essensiell}. For eks, la universet være $\mathbb{R}$: se på disse to: \[\forall n \exists m (n + m = 0), \exists m \forall n (n + m = 0)\] 
	\item 
	La her universet være $\mathbb{Z}$. Under er noen opppgaver fra \textbf{Eksamen V16}
	\item $\forall n \exists m (n^2 < m)$
	\item $\forall n \exists m (n + m = 0)$
	\item $\forall n \forall \exists k (2k = m + n)$
	\item $\forall a > 0 \forall b > 0 (ab = gcd(a,b) * lcm(a,b))$
\end{itemize}
\end{frame}

\begin{frame}
\frametitle{Hva er mengder?}
Ikke annet enn en samling av objekter. Grunnleggende mengder:

\begin{itemize}
	\item $\emptyset$ - den tomme mengden. $|\emptyset| = 0$.
	\item $\{\emptyset\}$ - mengden som inneholder $\emptyset$ (altså $\emptyset \in \{\emptyset\})$.
	\item $\mathbb{Z},\mathbb{Q},\mathbb{R},\mathbb{N}$ - heltall, rasjonelle tall, reelle tall, naturlige tall.
	\item \textit{Den viktigste ideen}; bygg nye mengder fra gamle: $\{x | x \in \mathbb{Q} \wedge x > 1\}$
	\item $A \cup B$ er mengden av ting som ligger i A og/eller i B. Altså, $x \in A \cup B \equiv x \in A \vee x \in B$.
\end{itemize}
\end{frame}

\begin{frame}
Se på oppgave 2 fra V16
\end{frame}

\subsection{Funksjoner}
\begin{frame}
	\frametitle{Funksjoner: Injeksjoner, surjeksjoner og bijeksjoner}
	En funksjon $f:A \rightarrow B$ er en "maskin" som tar inn ting i $A$ og spytter ut ting i $B$.
	
	\begin{enumerate}
		\item Injektiv: Hvis $(f(x) = f(y))$ så er $x = y$.
		\item Surjektiv: Hvis $b \in B$ så finnes det en $a \in A$ hvor $f(a) = b$.
		\item Bijektiv: Begge de to over
	\end{enumerate}
\end{frame}


\section{Litt tallteori}

\subsection{Deling}

\begin{frame}
	\frametitle{Deling - tilbake til barneskolen!}
	Hvordan regne ut $8/3$? Barneskole: $8/3 = 2$ med rest $2$. Ungdomskole: $8/3 = 2,666.. = 2,67$. Eksempel: $95/9 = 10\cdot9 + 5$. \linebreak
	
	Hvis resten er 0 "det går opp"; $x/d = qd + 0$, da sier vi at $d$ deler $x$, og vi skriver det $d|x$.
\end{frame}

\subsection{Primtall, gcd, lcm}

\begin{frame}
	\frametitle{Primtallfinning}
\end{frame}

\begin{frame}
	\frametitle{gcd og lcm}
	\begin{itemize}
		\item $\frac{5}{8} + \frac{3}{12}$
		\item $\text{lcm}(8,12) = 24$
		\item gcd og lcm - største felles faktor og minste felles multiplum.

	\end{itemize}
\end{frame}

\begin{frame}
	\frametitle{Euklids algoritme}
\end{frame}

\section{Induksjon}

\begin{frame}
\[\text{Del 4}\]
\end{frame}

\begin{frame}
\frametitle{Induksjon på naturlige tall}
Generelt prinsipp for å bevise påstander, ved å utnytte et konsept om størrelse: Begynn med de minste elementene, og jobb oppover. \linebreak

Noen eksempler av induksjon på naturlige tall:
\begin{itemize}
	\item $\forall n (n \geq 1 \rightarrow 1 + 3 + 5 + 7 +\dots + (2n-1) = n^2)$
	\item La $f_0 = 0, f_1 = 1$ og $f_n = f_{n-1} + f_{n-2}$ når $n \geq 2$. Vis at $f_n \geq (3/2)^{n-1}$ når $n \geq 6$.
\end{itemize}

\end{frame}

\begin{frame}
\frametitle{Induksjon på helt andre ting (strukturell induksjon)}
Bevis for at euklids algoritme fungerer for alle naturlige tall $>0$. Input er et par av tall, $(x,y)$. Vi kan ikke bare gjøre induksjon på x, eller y. 
\[P = \{(a,b) \in \mathbb{N}| a > 0 \wedge b > 0\}\]

Ide: Gjør induksjon på \textit{summen} av de to tallene!

\begin{enumerate} 
	\item Basis: Når summen er 2 (1,1) - da gir den rett svar.
	\item Induksjonshypotese: Anta at algoritmen gir rett svar når summen er mindre enn eller lik $k \geq 2$.
	\item Induksjons-steg: Når summen er lik $k+1$ har vi input $(a,b), a + b = k + 1$. Tre muligheter: \linebreak
	
	$a > b$ \\
	$b > a$ \\
	$a = b$
\end{enumerate}
\end{frame}

\begin{frame}
\frametitle{Telling i grafer}
Induktiv definisjon av grafer:

\begin{itemize}
	\item Basis: Den tomme grafen $\emptyset$ er en graf.
	\item Induksjonssteg 1: Vi kan legge inn en node i grafen.
	\item Induksjonssteg 2: Vi kan, mellom to punkter som ikke allerede har en linje, trekke en linje.
\end{itemize}

Graden til en node/punkt er antall kanter som er tegnet mellom den og andre punkter/node.
\linebreak
Vis: \textit{Summen av alle nodene sine grader er et partall}'

\end{frame}

\begin{frame}
\frametitle{Takk for meg!}
\end{frame}

\end{document}
